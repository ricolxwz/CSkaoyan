\part{操作系统}
\chapter{计算机系统概述}
\section{操作系统的基本概念}
\paragraph{03}a A.没有任何软件支持的计算机是裸机, 外面是操作系统, 操作系统提供资源管理功能和用户服务. 覆盖了软件的机器被称为扩充机, \textbf{管理计算机裸机是操作系统关心的问题}.\par D.\textbf{编译器实际上就是一段代码}, 操作系统只关心计算机中的文件的逻辑结构/物理结构/文件内部结构/多文件之间的组织问题, \textbf{不会关心文件的具体内容}.
\paragraph{05} C.并发性指的是若干事件在同一时间间隔(段)内发生.
\paragraph{08} C.系统调用是程序接口中的命令
\paragraph{13} A.shell是命令解释器, 命令接口.\par D.缓存全部是由系统管理的, 不提供用户接口
\section{操作系统发展历程}
\paragraph{07} B.操作系统追求的目标是及a时快速可靠响应, 不要求提高利用率.
\paragraph{09} A.加大时间片会增加系统的响应时间.\par B.静态页式管理: 预先给作业或者作业分配足够的内存空间, 和响应时间无关.\par C.优先级+非抢占式调度算法:可以让重要的作业得到及时响应, 有可以保证次要的作业得不到系统及时响应. \par D.代码可重入: 又叫作纯代码, 指一种允许多个进程同时访问的代码.
\paragraph{14} \uppercase\expandafter{\romannumeral3}.当一个进程因IO请求而暂停的时候, 系统就会通过中断执行另一个进程.
\section{操作系统运行环境}
\paragraph{01} A.通用操作系统是使用时间片轮转的系统, 无需预定运行时间.\par B.程序运行需要确定起始地址.\par C.高级程序设计语言的编译器在操作系统之上.
\paragraph{02} A.批处理的缺点是缺乏人机交互能力.\par B.IO指令是一定在核心态下运行的.\par C.批处理系统提升了系统利用率和吞吐量.\par D.通道是一种硬件技术.
\paragraph{06} D.操作系统可以使用特权指令和非特权指令, 用户程序只能使用非特权指令(与访管指令, 广义指令区分)
\paragraph{07} A.进程调度是由操作系统的调度算法实现的.\par B.时钟管理需要硬件计数器.\par C.地址映射需要基地址寄存器和地址加法器.\par D.系统终端的前三步需要硬件支持(隐指令)
\paragraph{08} C.当中断或者异常发生的时候, 会从用户态立即切换到核心态(由硬件完成), 在核心态下执行中断处理程序(中断处理程序属于操作系统程序).
\paragraph{18} C.进程切换一定会进入到内核态.
\paragraph{19} 
\chapter{进程与线程}
\section{进程与线程}
\paragraph{01} C.进程映像是PCB/数据段/程序段的组合, 是进程在某一时刻的快照.
\paragraph{03} C.进程之间传递数据的三种方式: 共享存储(基于数据结构和基于存储区)/消息传递(直接和间接)/管道通信.
\paragraph{05} C.可能全部程序都在死锁的状态, 即全部在阻塞态(重要思想).
\paragraph{08} C.有两种可能, 一是这10个进程全部都在阻塞态(注意是阻塞态, 不是就绪态), 二是有9个进程在就绪态, 1个进程在运行态, 故在就绪队列中最多有9个.
\paragraph{10} D.请求的IO操作完成, 说明请求的IO资源空闲, 进程可以由阻塞态转为就绪态.
\paragraph{12} D.程序封闭性的含义是程序的执行结果只与自身有关, 不受外界影响. 也就是说, \textbf{进程的执行速度不会影响执行的结果}, 如果程序不封闭的话, 程序的执行结果与执行速度有关(重要思想).
\paragraph{20} C语言编写的程序在使用内存的时候分为三段: 一是正文段, 用于存储代码和常量; 二是堆段, 用于存储动态分配的数据; 三是栈段, 用于存储临时分配的局部变量.
\paragraph{22} B.同一个系统的进程(或线程)可以被不同的进程(或线程)多次使用.










