\part{操作系统}
\chapter{计算机系统概述}
\section{操作系统的基本概念}
\paragraph{03} A.没有任何软件支持的计算机是裸机, 外面是操作系统, 操作系统提供资源管理功能和用户服务. 覆盖了软件的机器被称为扩充机, \textbf{管理计算机裸机是操作系统关心的问题}.\par D.\textbf{编译器实际上就是一段代码}, 操作系统只关心计算机中的文件的逻辑结构/物理结构/文件内部结构/多文件之间的组织问题, \textbf{不会关心文件的具体内容}.
\paragraph{05} C.并发性指的是若干事件在同一时间间隔(段)内发生.
\paragraph{08} C.系统调用是程序接口中的命令
\paragraph{13} A.shell是命令解释器, 命令接口.\par D.缓存全部是由系统管理的, 不提供用户接口
\section{操作系统发展历程}
\paragraph{07} B.操作系统追求的目标是及时快速可靠响应, 不要求提高利用率.
\paragraph{09} A.加大时间片会增加系统的响应时间.\par B.静态页式管理: 预先给作业或者作业分配足够的内存空间, 和响应时间无关.\par C.优先级+非抢占式调度算法:可以让重要的作业得到及时响应, 有可以保证次要的作业得不到系统及时响应. \par D.代码可重入: 又叫作纯代码, 指一种允许多个进程同时访问的代码.
\paragraph{14} \uppercase\expandafter{\romannumeral3}.当一个进程因IO请求而暂停的时候, 系统就会通过中断执行另一个进程.
